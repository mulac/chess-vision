\begin{appendices}

%
% The first appendix must be "Self-appraisal".
%
\chapter{Self-appraisal}

<This appendix should contain everything covered by the 'self-appraisal' criterion in the mark scheme. Although there is no length limit for this section, 2---4 pages will normally be sufficient. The format of this section is not prescribed, but you may like to organise your discussion into the following sections and subsections.>

\section{Critical self-evaluation}
Firstly found it to be a really fun project and enjoyed writing all the components that make a machine learning solution.  Seeing the end result was even better as 
I'm genuienly surprised at how well it worked, beating all the open source solutions I could find out there.  Learnt a lot about statistics which I never covered at A-Level.

\section{Personal reflection and lessons learned}
Surpised at how effective transfer learning is and the significance of it's place in the future.
The importance of experiment tracking for research.

\section{Legal, social, ethical and professional issues}

\subsection{Legal issues}
Legal issues could only really enter my project through the data collection and usage aspects.  By using my own data of games I played with no personal information 
I was able to completely avoid any legal issues.  As I am opening my data and solution to be opensource as I found that I really appreciated when others did so legal 
issues may also enter such as licensing.  Any datasets collected with my tool will be the complete responsibility of the person who captured that information and would 
need to be included in the license.

\subsection{Social issues}
The one social issue I can really see with this project is if it is used for cheating.  I also believe it would be trivial to train my solution on digital games such 
as those on chess.com as the data is abundant and would probably work very well.  This could lead to some bad actors using it to run analysis while playing the game 
and so would ruin the game for those who want to play against real humans.

\subsection{Ethical issues}
Personally, there weren't any serious ethical issues except those that I have already mentioned.  The real ethics come into when the tools and techniques I have used 
throughout this project are used for other purposes.  Deep learning is improving at a rapid rate and because of it's nature it's applicability knows really no bounds.
I believe these techniques can be used for immense good and therefore should be avoided, however, the negative impact it could have if not careful could disastrous and 
something I am very passionate about keeping a very close eye on.

\subsection{Professional issues}
If this solution were to be used in a professional setting, such as recording chess games in competitions, and mistakes were made by the model, then I could see some 
pretty bad issues.  I can't image Carlsen being very happy if he wanted to analyse his game but the inference application failed to record to PGN 100\% correctly. 
This was not the aim of the project and more measures and effort would need to be taken to avoid consequences in professional settings.

%
% Any other appendices you wish to use should come after "Self-appraisal". You can have as many appendices as you like.
%
\chapter{External Material}
<This appendix should provide a brief record of materials used in the solution that are not the student's own work. Such materials might be pieces of codes made available from a research group/company or from the internet, datasets prepared by external users or any preliminary materials/drafts/notes provided by a supervisor. It should be clear what was used as ready-made components and what was developed as part of the project. This appendix should be included even if no external materials were used, in which case a statement to that effect is all that is required.>




%
% Other appendices can be added here following the same pattern as above.
%



\end{appendices}
