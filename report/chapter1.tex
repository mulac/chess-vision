\chapter{Introduction and Background Research}

% You can cite chapters by using '\ref{chapter1}', where the label must
% match that given in the 'label' command, as on the next line.
\label{chapter1}

% Sections and sub-sections can be declared using \section and \subsection.
% There is also a \subsubsection, but consider carefully if you really need
% so many layers of section structure.
\section{Introduction}

Algorithms such as Deep Blue \cite{parikh1980adaptive}, AlphaZero \cite{} and more recently Player of Games\cite{}
have enabled computers to out smart the smartest humans at the game of Chess.
But all these algorithms are bound to the digital world, rendered useless when
competing against humans on a real board. This project aims to solve a major
component of this: vision.
We consider the vision problem for chess to be two-fold; what is the current board
state and where are all of the pieces?  With this information, in combination with the
previous algorithms and a robot arm, the computer is no longer bounded to the
digital world.

% Must provide evidence of a literature review. Use sections
% and subsections as they make sense for your project.
\section{Literature review}
When we ask ourselves: what is the hard part of Chess?  
One may think it's the stratergy, another may say anticipating what the opposition
is going to do next is the hardest part and the more statistical amoung us may formulate it
as choosing the most likely action for victory.  It is highly unlikely however, that one would say that it is locating
where all the pieces are in 3D space.

For computers this is amoung the hardest of the challenges.  
Make reference to humans huge allocation of resources to vision. \cite{}
Why is it so hard for computers then? It's an inverse problem. 
Compare to solving the decision problem (minimax).  
The statistal calculation of whether to trade Queen's or block with a pawn has now become trival.  

\subsection{A Short History of Computer Vision}
\subsubsection{Classical Techniques}
\subsubsection{Image Recognition}
\subsubsection{Object Detection}
But in most applications there will be many things we want to recognise in an image.
\subsubsection{Instance Segmentation}
Why bounding boxes are not enough.  What is segmentation? Instance segmentation and then different approaches with pros and cons, i.e. YOLO and RCNN.
\subsubsection{Adding More Dimensions}
The real world is not percepived in static 2d images.  How do we add an understanding of 3 dimensions and time in our computer vision models?
Important for localisation in the real world.  Important for understanding things like object permenance.


\subsection{Computer Vision for Chess}
Finding the chess board is the first challenge, the most common solution as in \cite{} 
is to use the `findChessboard` function in openCV for camera calibration.  This **how does 
this work**.  